\documentclass[NET,english,aspectratio=169,notitleframe]{tumbeamer}
%settings
\usepackage[utf8]{inputenc}
\usepackage{pgfplotstable}
\usepackage{marvosym}
\usepackage{filecontents}
\usepackage{packages}
\usepackage{beamermods}

\usepackage[backend=bibtex, style=ieee]{biblatex}

\usepackage{csquotes}
\usepackage{amssymb}% http://ctan.org/pkg/amssymb
\usepackage{pifont}% http://ctan.org/pkg/pifont
\usepackage{booktabs,caption}
\usepackage{threeparttable}
\newcommand{\cmark}{\textcolor{TUMDarkGreen}{\ding{51}}}%
\newcommand{\xmark}{\textcolor{TUMDarkRed}{\ding{55}}}%

\usetikzlibrary{calc}
\usetikzlibrary{arrows.meta}


% For lecture mode (use package option 'lecture'):
%\lecture[GRNVS]{Grundlagen Rechnernetze und Verteilte Systeme}
%\module{IN0010}
%\semester{SoSe\,2016}
%\assistants{Johannes Naab, Stephan Günther, Maurice Leclaire}

\usepackage{pgfplots}
\pgfplotsset{compat=newest}
\usepackage{tumcolor}
\usepackage{tumcolors}
\usepackage{textpos}


\usepackage{pgfpages}
\usepackage{ifthen}
% ============================================================================
% jobname solution
% ============================================================================
\newif\ifsolution%
\ifthenelse{\equal{\detokenize{notes}}{\jobname}}{%
\setbeameroption{show notes on second screen=bottom}
\setbeamercolor{note page}{bg=white, fg=black}
\setbeamercolor{note title}{bg=white!95!black, fg=black}
}{
}

%\xdefinecolor{orange}{cmyk}{0,0.65,0.95,0} 
%\xdefinecolor{dblue}{cmyk}{1,0.54,0.04,0.19}
%\xdefinecolor{blue}{cmyk}{1,0.43,0,0}  
%\xdefinecolor{lblue}{cmyk}{0.65,0.19,0.01,0.04}
%\xdefinecolor{green} {cmyk}{0.35,0,1,0.2} 
%xdefinecolor{yellow}{rgb}{1.00,0.71,0.00} 
%\colorlet{lgtorange}{orange!20} 
%\colorlet{lgtdblue}{dblue!20} 
%\colorlet{lgtblue}{blue!20} 
%\colorlet{lgtlblue}{lblue!20} 
%colorlet{lgtgreen}{green!20} 
%\colorlet{lgtyellow}{yellow!20}

\newcommand\Wider[2][3em]{%
\makebox[\linewidth][c]{%
  \begin{minipage}{\dimexpr\textwidth+#1\relax}
  \raggedright#2
  \end{minipage}%
  }%
}
\usepackage{cancel}
\usepackage{minted}


\addbibresource{lit.bib}


% For beamer mode (default):
 % jeder von dem wir hier irgendwas nehmen, alphabetisch sortiert
\author[Paul Emmerich]{\textbf{Paul Emmerich}, Simon Ellmann, Georg Carle}
\title{Safe and Secure User Space Drivers}
\date{February 28, 2019}

\makeatletter
\let\@@magyar@captionfix\relax
\makeatother

\begin{document}

\setbeamertemplate{footline}{}
  \begin{frame}[c,noframenumbering]
%    \begin{tikzpicture}[overlay,remember picture]
%      \node[opacity=0.5,anchor=south east] at ($(current page.south east)+(-1,-1)$) {%
%    \includegraphics[width=.4\textwidth]{pics/TUM_Uhrenturm.png}};
%  \end{tikzpicture}
  \centering%
  \Large%
  \strut\textcolor{TUMBlue}{\inserttitle}%
  \\[4ex]%
  \normalsize%
\footnotesize  \strut\insertauthor%
  \\[2ex]%
  \footnotesize%
  \insertdate%
  \\[4ex]%
  \ifdefined\departmentname%
    \ifdefined\chairname%
      \chairname\\%
    \fi%
    \departmentname\\%
  \fi%
  \TUMname\\%
\end{frame}
\setbeamertemplate{footline}[tumfootline]


\begin{frame}{Network drivers}
\centering\includegraphics[width=0.60\textwidth]{pics/nic3}\\
\vspace{-1em}\tiny{Intel XL710 [Picture: Intel.com]}
\end{frame}

\begin{frame}{Network driver complexity is increasing}
\centering\includegraphics[scale=1.1]{figures/drivers-loc-scatterplot}
\end{frame}


\begin{frame}{The ixy driver}
\begin{itemize}
\item Our attempt to write a simple yet fast user space network driver
\item It's a user space driver you can easily understand and read
\item Supports Intel ixgbe NICs (82599, X540, Xeon D, ...) and VirtIO 
\item $\approx$ 1,000 lines of C code, full of references to datasheets and specs
\item Intel driver: 38,000 lines in DPDK, 30,000 in Linux
\item Small code size makes it ideal for trustworthy systems
%\item Check it out on GitHub: \url{https://github.com/emmericp/ixy}
\vspace{1ex}
\item But is C the best language for drivers?
\end{itemize}
\end{frame}

%\begin{frame}{Expectation: Beautiful C code}
%\begin{itemize}
%\item Why write a driver in C?
%\pause
%\vspace{1em}
%\item Most drivers are written in C
%\item C is the lowest common denominator of systems programming languages
%\item C code can be beautiful
%\item Everyone can read C?
%\end{itemize}
%\end{frame}
%
%\begin{frame}[fragile]{Reality: C can be ugly}
%\begin{minted}[autogobble]{c}
%#define mystery_macro(ptr, type, member) ({\
%	const typeof(((type*)0)->member)* __mptr = (ptr);\
%	(type*)((char*)__mptr - offsetof(type, member));\
%})
%\end{minted}
%\end{frame}
%
%\begin{frame}[fragile]{Reality: C can be ugly}
%\begin{minted}[autogobble]{c}
%#define container_of(ptr, type, member) ({\
%	const typeof(((type*)0)->member)* __mptr = (ptr);\
%	(type*)((char*)__mptr - offsetof(type, member));\
%})
%\end{minted}
%\end{frame}
%
%
%\begin{frame}[fragile]{Reality: C can be ugly}
%\begin{minted}[autogobble]{c}
%#define container_of(ptr, type, member) ({\
%	const typeof(((type*)0)->member)* __mptr = (ptr);\
%	(type*)((char*)__mptr - offsetof(type, member));\
%})
%\end{minted}
%\begin{itemize}
%\item Allows some ``inheritance'' in C to abstract driver implementations
%\item Virtually all C drivers use this macro
%\item The Linux kernel contains $\approx$ 15,000 uses of this macro
%\end{itemize}
%\end{frame}
%
\begin{frame}{C can cause security problems}
\centering\includegraphics[trim={0 13cm 0 0},clip,width=0.65\textwidth]{pics/cve}

\footnotesize (...)

\centering\includegraphics[trim={0 0 0 17.5cm},clip,width=0.65\textwidth]{pics/cve}

\begin{itemize}
\item Screenshot from \url{https://www.cvedetails.com/}
\item Security bugs found in the Linux kernel in the last $\approx$ 20 years
\end{itemize}

\end{frame}


\begin{frame}{C can cause security problems}
\begin{itemize}
\item Not all bugs can be blamed on the language
\item Cutler et al. analyzed 65 CVEs categorized as code execution in the Linux kernel \footnote{C. Cutler, M. F. Kaashoek, and R. T. Morris, \emph{``The benefits and costs of writing a POSIX kernel in a high-level language''}, USENIX OSDI, 2018}
\end{itemize}
\pause
\begin{table}
\centering
\begin{tabular}{ l  r r l }
  \toprule
  Bug type & Num. & Perc. & Can be avoided by a high-level language? \\
  \midrule
  Various & 11 & 17\% & Unclear/Maybe \\
  Logic & 14 & 22\% & No \\
  Use-after-free & 8 & 12\% & Yes \\
  Out of bounds & 32 & 49\% & Yes (likely leads to panic) \\
  \bottomrule  
\end{tabular}
\caption{Code execution vulnerabilities in the Linux kernel identified by Cutler et al$^1$}
\end{table}

\end{frame}

\begin{frame}{Are there preventable bugs in drivers?}
\begin{itemize}
\item We looked at these 40 preventable bugs
\pause
\item 39 of them were in drivers (the other was in the Bluetooth stack)
\end{itemize}
\end{frame}

\begin{frame}{Should drivers for trustworthy systems be written in C?}
\begin{itemize}
\item If you have a choice: probably not
\pause
\item User space drivers can be written in \emph{any} language!
\item But are all languages an equally good choice?
\item Is a JIT compiler or a garbage collector a problem in a driver?
\end{itemize}
\end{frame}

%\setbeamertemplate{footline}{}
%\setbeamertemplate{headline}{}
%\begin{frame}{}
%\centering\includegraphics[width=0.65\textwidth]{pics/allthe1}
%\end{frame}

%\begin{frame}{}
%\centering\includegraphics[width=0.65\textwidth]{pics/allthe2}
%\end{frame}

%\begin{frame}{}
%\centering\includegraphics[width=0.85\textwidth]{pics/theses}
%\end{frame}
%\setbeamertemplate{headline}[tumheadline]
%\setbeamertemplate{footline}[tumfootline]


%\begin{frame}{Basics: How to talk to (modern) PCIe devices}
%\begin{enumerate}
%\item Memory-mapped IO (MMIO)
%\item Direct memory access (DMA)
%\item Interrupts
%\end{enumerate}
%\end{frame}

%\begin{frame}{Basics: How to talk to (modern) PCIe devices}
%\begin{enumerate}
%\item Memory-mapped IO (MMIO)
%\begin{itemize}
%\item Magic memory area that is mapped to the device
%\item Memory reads/writes are directly forwarded to the device
%\item Usually used to expose device registers
%\item User space drivers: \texttt{mmap} a magic file
%\end{itemize}
%\item[\color{TUMLightGray}2.] {\color{TUMLightGray} Direct memory access (DMA)}
%\item[\color{TUMLightGray}3.] {\color{TUMLightGray} Interrupts}
%\end{enumerate}
%\end{frame}

%\begin{frame}{Basics: How to talk to (modern) PCIe devices}
%\begin{enumerate}
%\item[\color{TUMLightGray}1.] {\color{TUMLightGray} Memory-mapped IO (MMIO)}
%\item[2.] Direct memory access (DMA)
%\begin{itemize}
%\item Allows the device to read/write \emph{arbitrary} memory locations
%\item User space drivers: figure out physical addresses, tell the device to write there
%\end{itemize}
%\item[\color{TUMLightGray}3.] {\color{TUMLightGray}  Interrupts}
%\end{enumerate}
%\end{frame}

%\begin{frame}{Basics: How to talk to (modern) PCIe devices}
%\begin{enumerate}
%\item[\color{TUMLightGray}1.] {\color{TUMLightGray} Memory-mapped IO (MMIO)}
%\item[\color{TUMLightGray}2.] {\color{TUMLightGray} Direct memory access (DMA)}
%\item[3.] Interrupts
%\begin{itemize}
%\item This is how the device informs you about events
%\item User space drivers: available via the Linux \texttt{vfio} subsystem
%\item (Usually) not useful for high-speed network drivers 
%\item We'll ignore interrupts here
%\end{itemize}
%\end{enumerate}
%\end{frame}


%\begin{frame}{How to write a user space driver in 4 simple steps}
%\begin{itemize}
%\item[1.] Unload kernel driver
%\item[2.] \texttt{mmap} the PCIe MMIO address space
%\item[3.] Figure out physical addresses for DMA
%\item[4.] Write the driver
%\end{itemize}
%\end{frame}

%\begin{frame}{Hardware: Intel \texttt{ixgbe} family (10\,Gbit/s)}
%\begin{itemize}
%\item \texttt{ixgbe} family: 82599ES (aka X520), X540, X550, Xeon D embedded NIC
%\item Commonly found in servers or as on-board chips
%\item Very good datasheet publicly available
%\vspace{1em}
%\item Almost no logic hidden behind black-box firmware
%%\item<2-> Black-box firmware contains almost no magic
%\item<2-> Drivers for many newer NICs often just exchanges messages with the firmware
%\item<2-> Here: all hardware features directly exposed to the driver
%\end{itemize}
%\end{frame}



\newmintinline[ccode]{c}{}
\newmintinline[bashcode]{c}{}

%\begin{frame}[fragile=singleslide]{Find the device we want to use}
%\begin{Verbatim}[commandchars=\\\{\}]
%# lspci
%03:00.0 Ethernet controller: Intel Corporation 82599ES 10-Gigabit SFI/SFP+ ...
%03:00.1 Ethernet controller: Intel Corporation 82599ES 10-Gigabit SFI/SFP+ ...
%\end{Verbatim}
%\end{frame}
%
%\begin{frame}[fragile=singleslide]{Find the device we want to use}
%\begin{Verbatim}[commandchars=\\\{\}]
%# lspci
%\textbf{03:00.0} Ethernet controller: Intel Corporation 82599ES 10-Gigabit SFI/SFP+ ...
%\textbf{03:00.1} Ethernet controller: Intel Corporation 82599ES 10-Gigabit SFI/SFP+ ...
%\end{Verbatim}
%\end{frame}
%
%\begin{frame}[fragile=singleslide]{Unload the kernel driver}
%\begin{minted}{bash}
%echo 0000:03:00.1 > /sys/bus/pci/devices/0000:03:00.1/driver/unbind
%\end{minted}
%\end{frame}
%
%\begin{frame}[fragile=singleslide]{\texttt{mmap} the PCIe register address space from user space}
%\begin{minted}[autogobble]{c}
%int fd = open("/sys/bus/pci/devices/0000:03:00.0/resource0", O_RDWR);
%struct stat stat;
%fstat(fd, &stat);
%uint8_t* registers = (uint8_t*) mmap(NULL, stat.st_size, PROT_READ | PROT_WRITE,
%                                     MAP_SHARED, fd, 0);
%\end{minted}
%\end{frame}
%
%\begin{frame}{Device registers}
%\centering\includegraphics[width=0.75\textwidth]{pics/registers}
%\end{frame}
%
%\begin{frame}[fragile=singleslide]{Access registers: LEDs}
%\begin{minted}[autogobble]{c}
%#define LEDCTL 0x00200
%#define LED0_BLINK_OFFS 7
%
%uint32_t leds = *((volatile uint32_t*)(registers + LEDCTL));
%*((volatile uint32_t*)(registers + LEDCTL)) = leds | (1 << LED0_BLINK_OFFS);
%\end{minted}
%\begin{itemize}
%\item Memory-mapped IO: all memory accesses go directly to the NIC
%\item One of the very few valid uses of \ccode{volatile} in C
%\end{itemize}
%\end{frame}


%\begin{frame}{Handling packets via DMA}
%\begin{itemize}
%\item Packets are transferred via queue interfaces (often called rings)
%\item Rings are configured via MMIO and accessed by the device via DMA
%\item Rings (usually) contain pointers to packets, also accessed via DMA
%\pause
%\vspace{1em}
%\item Details vary between different devices
%\item This is not unique to NICs: most PCIe devices work in a similar manner
%\end{itemize}
%\end{frame}


%\begin{frame}{Challenges for high-level languages}
%\begin{itemize}
%\item Access to \texttt{mmap} with the proper flags
%\item Handle externally allocated (foreign) memory in the language
%\item Handle memory layouts/formats (i.e., access memory that looks like a given C struct)
%\item Memory access semantics: memory barriers, volatile reads/writes
%\item Some operations in drivers are inherently unsafe
%\end{itemize}
%\end{frame}




\begin{frame}{We wrote full user space drivers in these languages}
\begin{figure}
    \centering

    \begin{subfigure}[t]{0.2\textwidth}
        \centering
        \scalebox{1.5}{\Huge C\#}
    \end{subfigure}
    ~ 
    \begin{subfigure}[t]{0.3\textwidth}
        \centering
        \includegraphics[width=0.8\textwidth]{pics/swift}
    \end{subfigure}
    ~ 
    \begin{subfigure}[t]{0.3\textwidth}
        \centering
        \includegraphics[width=0.9\textwidth]{pics/ocaml}
    \end{subfigure}
    \\
    \vspace{2.5em}
    \centering
    ~ 
    \begin{subfigure}[t]{0.2\columnwidth}
        \centering
        \includegraphics[width=0.5\textwidth]{pics/haskell}
    \end{subfigure}
    ~ 
    \begin{subfigure}[t]{0.2\columnwidth}
        \centering
        \includegraphics[width=0.95\textwidth]{pics/go}
    \end{subfigure}
    ~ 
    \begin{subfigure}[t]{0.2\columnwidth}
        \centering
        \includegraphics[width=0.45\textwidth]{pics/rust}
    \end{subfigure}
    ~ 
    \begin{subfigure}[t]{0.3\columnwidth}
	\centering
        \includegraphics[width=0.9\textwidth]{pics/python}
    \end{subfigure}
\end{figure}
\end{frame}

\begin{frame}{Goals for our implementations}
\begin{itemize}
\item Implement the same feature set as our C reference driver
\item Use a similar structure like the C driver
\item Write idiomatic code for the selected language
\item Use language safety features where possible
\item Quantify trade-offs for performance vs. safety
\vspace{1em}
\item This allows us to compare different languages for safety-critical systems
\end{itemize}
\end{frame}


\begin{frame}{Language comparison: Overview}
\begin{table}[t]
 \setlength{\tabcolsep}{2mm}
	\centering
	\footnotesize
	\begin{tabular}{lrrrr}
		\textbf{Language} & \textbf{Main paradigm} & \textbf{Memory management} & \textbf{Compilation} \\
		\toprule
		Rust & Imperative & Ownership/RAII & (LLVM) Compiled \\
		Go & Imperative & Garbage collection & Compiled \\
		C\# & Object-oriented & Garbage collection & JIT \\
		Swift & Protocol-oriented & Reference counting & (LLVM) Compiled \\
		OCaml & Functional & Garbage collection & Compiled \\
		Haskell & Functional & Garbage collection & (LLVM) Compiled \\
		Python & Imperative & Garbage collection & Interpreted \\
		\bottomrule
	\end{tabular}
	\caption{Language overview}
	\label{tbl:languages}
\end{table}
\end{frame}

\setbeamertemplate{footline}{}
\begin{frame}{Language comparison: Safety properties}
\begin{table}[t]
 \setlength{\tabcolsep}{1.3mm}
	\centering
	\footnotesize
	\begin{tabular}{lccccc}
		& \multicolumn{2}{c}{\textbf{General memory}} & \multicolumn{2}{c}{\hspace{-1em}\textbf{Packet buffers}}  \\
		\textbf{Language} & \textbf{Bounds checks} & \textbf{Use after free}  & \textbf{Bounds checks} & \textbf{Use after free} & \textbf{Int overflows} \\
		\toprule
		C & \xmark & \xmark & \xmark & \xmark & \xmark \\
		Rust & \cmark & \cmark & (\cmark)$^1$ & \cmark & (\cmark)$^4$ \\
		Go & \cmark & \cmark & (\cmark)$^1$ & (\cmark)$^3$ & \xmark \\
		C\# & \cmark & \cmark & (\cmark)$^1$ & (\cmark)$^3$ & \xmark \\
		Swift & \cmark & \cmark & \xmark$^2$ & (\cmark)$^3$ & \cmark \\
		Haskell & \cmark & \cmark & (\cmark)$^1$ & (\cmark)$^3$ & \xmark \\
		OCaml & \cmark & \cmark & (\cmark)$^1$ & (\cmark)$^3$ & \xmark \\
		Python & \cmark & \cmark & (\cmark)$^1$ & (\cmark)$^3$ & \xmark \\
		\bottomrule
	\end{tabular}
	\begin{tablenotes}
	\item $^1$ Bounds enforced by wrapper, constructor in unsafe code
	\item $^2$ Bounds only enforced in debug mode
	\item $^3$ Buffers are never free'd, only returned to a memory pool
	\item $^4$ Disabled by default, proposed to be enabled by default in the future
	\end{tablenotes}
	\caption{Language-level protections against classes of bugs in our drivers}
	\label{tbl:lang-safety}
	\vspace{-3em}
\end{table}
\end{frame}
\setbeamertemplate{footline}[tumfootline]


\begin{frame}{Language comparison: Implementation sizes}
\begin{table}[t]
 \setlength{\tabcolsep}{2mm}
	\centering
	\footnotesize
	\begin{tabular}{lrrr}
		\textbf{Lang.} & \textbf{Lines of code}$^1$ & \textbf{Lines of C code}$^1$  & \textbf{Code size (gzip$^2$)} \\
		\toprule
		C & 831 & 831 & 12.9\,kB  \\
		Rust & 961 & 0 & 10.4\,kB\\
		Go & 1640 & 0 & 20.6\,kB \\
		C\# & 1266 & 34 & 13.1\,kB\\
		Swift & 1506 & 0 & 15.9\,kB \\
		Haskell & 1001 & 0 &  9.6\,kB\\
		OCaml & 1177 & 28 &  12.3\,kB\\
		Python & 1242 & (Cython) 77 &14.2\,kB \\
		\bottomrule
	\end{tabular}
	\begin{tablenotes}
	\item $^1$ Excluding empty lines and comments, counted with \texttt{cloc}
	\item $^2$ Compression level 6
	\end{tablenotes}
	\caption{Size of our implementations (w/o register offset constants, stripped features not found in all drivers)}
	\label{tbl:lang-lines}
	\vspace{-3em}
\end{table}
\end{frame}



%\setbeamertemplate{footline}{}
%\setbeamertemplate{headline}{}
%\setbeamertemplate{headline}[tumheadline]
%\setbeamertemplate{footline}[tumfootline]
%\begin{frame}{}
%\centering \scalebox{4}{\Huge C\#}
%\end{frame}
%
%
%\begin{frame}{C\#}
%\begin{itemize}
%\item No, we didn't develop a Windows driver
%\item We used Microsoft's CoreCLR on Linux
%\pause
%\vspace{1em}
%\item JIT compiled
%\item Garbage collected
%\item Memory safe (mostly)
%\pause
%\vspace{1em}
%\item C\# supports a relatively obscure \emph{unsafe mode}
%\item Unsafe mode features full support for pointers
%\end{itemize}
%\end{frame}
%
%\begin{frame}[fragile]{C\#: Access to external memory}
%\begin{itemize}
%\item C\# provides \texttt{UnmanagedMemoryStream}, a nice wrapper for foreign memory
%\item But it was too slow :(
%\pause
%\item Use unsafe raw pointers for packet buffers instead
%\end{itemize}
%\begin{minted}[autogobble]{csharp}
%public unsafe void WriteData(uint offset, int val) {
%    if (offset >= BUF_SIZE) throw new IndexOutOfRangeException();
%    volatile int *ptr = (volatile int*)(_baseAddress + DataOffset + offset);
%    *ptr = val;
%}
%\end{minted}
%\begin{itemize}
%\pause
%\item Looks a lot like C
%\item Potentially unsafe operations are all in a few known places, simpler auditing
%\end{itemize}
%\end{frame}
%
%
%%\begin{frame}[]{C\#: Summary}
%%\begin{itemize}
%%\item ???
%%\end{itemize}
%%\end{frame}
%
%\setbeamertemplate{footline}{}
%\setbeamertemplate{headline}{}
%\begin{frame}{}
%\centering\includegraphics[width=0.8\textwidth]{pics/swift}
%\end{frame}
%\setbeamertemplate{headline}[tumheadline]
%\setbeamertemplate{footline}[tumfootline]
%
%\begin{frame}{Swift}
%\begin{itemize}
%\item No, we didn't develop a macOS/iOS driver
%\item Swift is also available on Linux
%\pause
%\vspace{1em}
%\item Compiled via LLVM
%\item Memory management via reference counting (ARC)
%\item Memory safe (mostly)
%%\pause
%%\vspace{1em}
%%\item Access to external memory via \texttt{UnsafeBufferPointer} 
%\end{itemize}
%\end{frame}
%
%\begin{frame}[fragile]{Swift: Pointers}
%\begin{itemize}
%\item \texttt{UnsafeBufferPointer} and co wrap foreign memory blobs
%\item Used to make packets in DMA buffers available
%\end{itemize}
%\begin{minted}[autogobble]{swift}
%public var packetData: UnsafeBufferPointer<UInt8>? {
%    get {
%        return UnsafeBufferPointer<UInt8>(
%            start: self.entry.pointer.assumingMemoryBound(to: UInt8.self),
%            count: Int(self.size)
%        )
%    }
%}
%\end{minted}
%\begin{itemize}
%\item Forces you to specify the buffer size, accesses check the bounds in debug mode
%\end{itemize}
%\end{frame}
%
%\begin{frame}[fragile]{Swift: Pointers with a very verbose syntax}
%\begin{itemize}
%\item Example: modify one byte in a packet (part of our benchmark)
%\end{itemize}
%\begin{minted}[autogobble]{swift}
%public func touch() {
%    let ptr = self.entry.pointer
%    var newValue: UInt32 = ptr.load(fromByteOffset: 0, as: UInt32.self)
%    newValue += 1
%    ptr.storeBytes(of: newValue, toByteOffset: 0, as: UInt32.self)
%}
%\end{minted}
%\begin{itemize}
%\item Quite verbose compared to C or C\#
%\end{itemize}
%\end{frame}
%
%
%
%\setbeamertemplate{footline}{}
%\setbeamertemplate{headline}{}
%\begin{frame}{}
%\centering\includegraphics[width=0.8\textwidth]{pics/ocaml}
%\end{frame}
%\setbeamertemplate{headline}[tumheadline]
%\setbeamertemplate{footline}[tumfootline]
%
%\begin{frame}{OCaml}
%\begin{itemize}
%\item Compiled language
%\item Memory management via garbage collection
%\item Memory safe
%\item Functional language
%\end{itemize}
%\end{frame}
%
%
%\begin{frame}[fragile]{OCaml: Cstruct}
%\begin{minted}[autogobble]{ocaml}
%[%%cstruct
%  type adv_rxd_wb = {
%    pkt_info : uint16;
%    hdr_info : uint16;
%    ip_id : uint16;
%    csum : uint16;
%    status_error : uint32;
%    length : uint16;
%    vlan : uint16
%  } [@@little_endian]
%]
%\end{minted}
%\begin{itemize}
%\item Cstruct generates accessors to work with (foreign) memory that looks like this
%\end{itemize}
%\end{frame}
%
%\begin{frame}[fragile]{OCaml: It looks quite different}
%\begin{itemize}
%\item Code that checks how many packets are ready to be read in the receive ring
%\end{itemize}
%\begin{minted}[autogobble]{ocaml}
%let num_done =
%  (* counting without mutation *)
%  let rec loop offset =
%    let rxd = descriptors.(wrap_rx (rxq.rx_index + offset)) in
%    if Int32.((get_adv_rx_wb_status rxd) land RXD.stat_dd <> 0l) then
%      loop (offset + 1)
%    else
%      offset in
%  loop 0
%\end{minted}
%\end{frame}
%
%
%%\begin{frame}[fragile]{OCaml: It looks quite different}
%%\begin{itemize}
%%\item Example of the forwarder application on top of the driver
%%\end{itemize}
%%\begin{minted}[autogobble]{ocaml}
%%let forward rx_dev tx_dev =
%%  let rx = Ixy.rx_batch rx_dev 0 in
%%  Ixy.tx_batch_busy_wait tx_dev 0 rx
%%\end{minted}
%%\end{frame}
%
%
%\setbeamertemplate{footline}{}
%\setbeamertemplate{headline}{}
%\begin{frame}{}
%\centering\includegraphics[width=0.6\textwidth]{pics/haskell}
%\end{frame}
%\setbeamertemplate{headline}[tumheadline]
%\setbeamertemplate{footline}[tumfootline]
%
%\begin{frame}{Haskell}
%\begin{itemize}
%\item Compiled language (GHC)
%\item Memory management via garbage collection
%\item Memory safe
%\item Functional language
%\end{itemize}
%\end{frame}
%
%
%\begin{frame}{Haskell: Access to foreign memory}
%\begin{itemize}
%\item \texttt{mmap} and \texttt{mlock} available via \texttt{System.Posix.Memory}
%\begin{itemize}
%\item All necessary flags and features are available in Haskell, we had to write some C code to get \texttt{mmap/mlock} in OCaml
%\end{itemize}
%\vspace{1em}
%\item \texttt{Foreign} package provides access to foreign memory
%\end{itemize}
%\end{frame}
%
%
%\begin{frame}[fragile]{Haskell: Sum types are useful}
%\begin{itemize}
%\item Descriptors often exist in two forms
%\begin{itemize}
%\item One format written by the driver and read by the device
%\item A second format that is written back by the device once it's finished
%\end{itemize}
%\end{itemize}
%\begin{minted}{haskell}
%data TransmitDescriptor = TransmitRead { tdBufPhysAddr :: !Word64
%                                       , tdCmdTypeLen :: !Word32
%                                       , tdOlInfoStatus :: !Word32 }
%                          | TransmitWriteback { tdStatus :: !Word32 }
%\end{minted}
%\end{frame}
%
%
%%\begin{frame}[fragile]{Haskell: How to write a TX descriptor}
%%\begin{minted}{haskell}
%%unsafeUseAsCStringLen pkt (\(ptr, len) -> copyBytes bufPtr (castPtr ptr) len)
%%poke descPtr TransmitRead { tdBufPhysAddr = bufPhysAddr 
%%                          , tdCmdTypeLen = fromIntegral $ cmdTypeLen size
%%                          , tdOlInfoStatus = fromIntegral $ shift size 14 }
%%writeIORef (txqIndexRef queue) nextIndex
%%\end{minted}
%%\end{frame}
%
%
%
%
%\setbeamertemplate{footline}{}
%\setbeamertemplate{headline}{}
%\begin{frame}{}
%\centering\includegraphics[width=0.8\textwidth]{pics/go}
%\end{frame}
%\setbeamertemplate{headline}[tumheadline]
%\setbeamertemplate{footline}[tumfootline]
%
%\begin{frame}{Go}
%\begin{itemize}
%\item Compiled programming language developed by Google 
%\item General purpose language but designed for distributed systems
%\item<2-> A driver is not a distributed system
%\item<3-> Then why even use Go?
%\begin{itemize}
%\item<4-> Runtime for:\\Garbage Collection\\Memory \& Type safety
%%\item<4-> Simple yet powerful concurrency via goroutines	%not relevant ofr a single core driver
%\item<4-> Large standard library
%\end{itemize}
%\end{itemize}
%\end{frame}
%
%\begin{frame}{Go for drivers}
%\begin{itemize}
%\item Actually a lot like C in many aspects
%%maybe include some code examples?
%\item<2-> Main differences:
%\begin{itemize}
%\item<2-> No pointer arithmetic (managing DMA memory)
%\item<2-> No volatile (memory barriers for register access)
%\end{itemize}
%\item<3-> What we do instead:
%\begin{itemize}
%\item<3-> Manage DMA memory via slices
%\item<3-> Unsafe pointers: circumvent runtime but allow arbitrary pointer\\
%	$\rightarrow$ Physical address calculation \& register access
%\item<3-> Rule set for unsafe pointers to still be valid
%\end{itemize}
%\end{itemize}
%\end{frame}
%
%%\begin{frame}[fragile]{Managing memory: mempools}
%%\begin{itemize}
%%\item syscall.Mmap() returns slice of the mmapped memory area
%%\item For this presentation: slice = fancy array\\
%%	$\rightarrow$ bounds checked, subslicing, etc.
%%\end{itemize}
%%%\end{frame}
%%
%%%\begin{frame}[fragile]{func MemoryAllocateMempool}
%%%mempool alloc code
%%\begin{minted}[autogobble]{go}
%%//allocate DMA memory & initialize mempool
%%for i := uint32(0); i < numEntries; i++ {
%%	mempool.packetBuf[i] = &PktBuf{
%%		Pkt:        mempool.buf[i*entrySize : (i+1)*entrySize],
%%		PhyAddr:    uint64(virtToPhys(uintptr(
%%			    unsafe.Pointer(&mempool.buf[i*entrySize])))),
%%	}
%%}
%%\end{minted}
%%\end{frame}
%
%\begin{frame}[fragile]{No volatile, no problem}
%%insert code registers
%\begin{itemize}
%\item Registers share memory with NIC
%\item Compiler memory barrier to prevent re-ordering
%\item sync/atomic functions prevent re-ordering around them
%\end{itemize}
%\begin{minted}[autogobble]{go}
%func setReg32(addr []byte, reg int, value uint32) {
%	atomic.StoreUint32((*uint32)(unsafe.Pointer(&addr[reg])), value)
%}
%
%func getReg32(addr []byte, reg int) uint32 {
%	return atomic.LoadUint32((*uint32)(unsafe.Pointer(&addr[reg])))
%}
%\end{minted}
%\end{frame}

%\begin{frame}{Conclusion Go}
%\begin{itemize}
%\item Actually quite nice to work with
%\begin{itemize}
%\item Safety (see Cutler et al.\footnote{C. Cutler, M. F. Kaashoek, and R. T. Morris, \emph{``The benefits and costs of writing a POSIX kernel in a high-level language''}, USENIX OSDI, 2018})
%\item Looks like C in beautiful
%\end{itemize}
%\item<2-> But
%\begin{itemize}
%\item<2-> Approx 10\% slower then C % at optimum batch size
%%\item<2-> Needs higher batch sizes to be efficient
%\item<2-> Descriptor access can be ugly (functions on descriptors are too costly)
%\end{itemize}
%\end{itemize}
%\end{frame}

%\setbeamertemplate{footline}{}
%\setbeamertemplate{headline}{}
%\begin{frame}{}
%\centering\includegraphics[height=0.87\textheight]{pics/rust}
%\end{frame}
%\setbeamertemplate{headline}[tumheadline]
%\setbeamertemplate{footline}[tumfootline]

%\begin{frame}{Rust}
%\emph{What is Rust?}\\
%\begin{quote}
%A safe, concurrent, practical systems language.
%\end{quote}\\
%\begin{itemize}
%\item No garbage collector
%\item Unique ownership system and rules for moving/borrowing values
%\item Unsafe mode
%\end{itemize}
%\end{frame}

%\begin{frame}{Safety in Rust: The ownership system}
%\begin{itemize}
%\item Immutability of variables by default
%\item Three rules:
%\begin{enumerate}
%\item Each value has a variable that is its owner
%\item There can only be one owner at a time
%\item When the owner goes out of scope, the value is freed
%\end{enumerate}
%\item Rules enforced at compile-time
%\item Ownership can be passed to another variable% by
%\begin{itemize}
%\item ``moving'' the value or by
%\item ``borrowing'' it through a reference
%\end{itemize}
%\end{itemize}
%\end{frame}

%\begin{frame}[fragile]{Safety in Rust: The ownership system by example}
%\begin{itemize}
%\item \texttt{Packet}s are owners of some DMA memory
%\item \texttt{Packet}s are passed between user code and the driver, thus ownership is passed as well
%\item At any point in time there is only one \texttt{Packet} owner that can change its memory
%\end{itemize}
%\begin{minted}{rust}
%let buffer: &mut VecDeque<Packet> = VecDeque::new();
%dev.rx_batch(RX_QUEUE, buffer, BATCH_SIZE);
%for p in buffer.iter_mut() {
%  p[48] += 1;
%}
%dev.tx_batch(TX_QUEUE, buffer);
%buffer.drain(..);
%\end{minted}
%\end{frame}

%\begin{frame}[fragile]{Safety in Rust: Unsafe code}
%\begin{itemize}
%\item Not everything can be done in safe Rust
%\item Calling foreign functions and dereferencing raw pointers is unsafe
%\item Many functions in Rust's standard library make use of unsafe code
%\end{itemize}
%\begin{minted}{rust}
%let ptr = unsafe {
%  libc::mmap(
%    ptr::null_mut(), len, libc::PROT_READ | libc::PROT_WRITE,
%    libc::MAP_SHARED, file.as_raw_fd(), 0,
%  ) as *mut u8
%};
%\end{minted}
%\end{frame}

%\begin{frame}[fragile]{Example: Setting registers}
%\begin{itemize}
%\item Biggest challenge: safe memory handling with unsafe code
%\end{itemize}
%\begin{minted}{rust}
%fn set_reg32(&self, reg: u32, val: u32) {
%  assert!(
%    reg as usize <= self.len - 4 as usize,
%    "memory access out of bounds"
%  );
%
%  unsafe {
%    ptr::write_volatile(
%        (self.addr as usize + reg as usize) as *mut u32, val
%    );
%  }
%}
%\end{minted}
%\end{frame}

%\setbeamertemplate{footline}{}
%\setbeamertemplate{headline}{}

\begin{frame}{Performance comparison: Test setup}
%\vspace{-.75cm}
\centering\includestandalone[scale=0.55]{figures/testsetup}
\end{frame}
\setbeamertemplate{footline}[tumfootline]
\setbeamertemplate{headline}[tumheadline]


%\begin{frame}{Performance comparison (single CPU core)}
%\centering\includestandalone[scale=0.8]{figures/benchmarks-all-throughput}
%\end{frame}

\begin{frame}{Batching at 3.3\,GHz CPU speed}
\centering\includegraphics[scale=1]{figures/batches-33.pdf}
\end{frame}

%\setbeamertemplate{footline}{}
%\setbeamertemplate{headline}{}
%\begin{frame}{Swift: Flame graph}
%\hspace{-.5cm}\includegraphics[width=1.065\textwidth]{pics/flamegraph}
%\end{frame}
%\setbeamertemplate{footline}[tumfootline]
%\setbeamertemplate{headline}[tumheadline]
%
%\begin{frame}{Swift: Why so slow?}
%\begin{itemize}
%\item Lots of time spent in Swift's memory management
%\item Swift adds calls to release/retain for each used object in each function
%\item This is basically the same as wrapping every object in a \texttt{std::shared\_ptr} in C++
%\vspace{1em}
%\pause
%\item Time in release/retain: 76\%
%\item For comparison: Go spends less than 0.5\% in the garbage collector
%\end{itemize}
%\end{frame}

%\begin{frame}{Garbage collection and JIT compilation vs. latency}
%\centering\includestandalone[scale=0.8]{figures/latency-cdf1}
%\end{frame}
%
%\begin{frame}{Garbage collection and JIT compilation vs. latency}
%\centering\includestandalone[scale=0.8]{figures/latency-cdf2}
%\end{frame}
%
%\begin{frame}{Garbage collection and JIT compilation vs. latency}
%\centering\includestandalone[scale=0.8]{figures/latency-cdf3}
%\end{frame}
%
%\begin{frame}{Garbage collection and JIT compilation vs. latency}
%\centering\includestandalone[scale=0.8]{figures/latency-cdf4}
%\end{frame}
%
%\begin{frame}{Complementary cumulative distribution function}
%\centering\includestandalone[scale=0.8]{figures/latency-ccdf}
%\end{frame}

\begin{frame}{Tail latency at 1\,Mpps}
\centering\includegraphics[scale=1.1]{figures/latency-1/latency-ccdf.pdf}
\end{frame}

\begin{frame}{Tail latency at 10\,Mpps}
\centering\includegraphics[scale=1.1]{figures/latency-10/latency-ccdf.pdf}
\end{frame}

\begin{frame}{Tail latency at 20\,Mpps}
\centering\includegraphics[scale=1.1]{figures/latency-20/latency-ccdf.pdf}
\end{frame}


\begin{frame}{Languages for code in trustworthy systems}
\begin{itemize}
\item Rust
\begin{itemize}
\item Fast, no garbage collector
\item Low-level: Easy to reason about performance
\item Safest language of the evaluated languages
\end{itemize}
\item Go
\begin{itemize}
\item Fast, low-latency garbage collector
\item Garbage collector tuned for sub-millisecond latency
\item Easier and faster to write than Rust
\end{itemize}
\pause
\item Other languages
\begin{itemize}
\item Implement critical parts in different languages in redundant systems
\item Functional languages for easier formal verification
\end{itemize}
\end{itemize}
\end{frame}


\begin{frame}{Conclusions}
\begin{itemize}
\item High-level languages can prevent entire classes of bugs
\item High-level languages are suitable for low-level code
\item Drivers are becoming more and more complex, simpler drivers reduce attack surface
\item Future work: A full stack in Rust (ixy + smoltcp), evaluating Redox
\vspace{1em}
\item Paper about safer drivers under submission to SIGCOMM
\item Code for all drivers available on GitHub: \\\url{https://github.com/ixy-languages/ixy-languages}

\end{itemize}
\end{frame}

\begin{frame}{Backup: Unprivileged user space drivers}
\begin{itemize}
\item User space drivers usually run with root privileges, but why?
\pause
\vspace{1em}
\item Mapping PCIe resources requires root
\item Allocating non-transparent huge pages requires root
\item Locking memory requires root
\vspace{1em}
\item Can we do that in a small separate program that is easy to audit and then drop privileges?
\pause
\item Yes, we can
\item But it's not really secure
\end{itemize}
\end{frame}

\begin{frame}{Memory access on modern systems}
\centering\includestandalone[scale=.70]{figures/iommu1}
\end{frame}

\begin{frame}{Memory access on modern systems}
\centering\includestandalone[scale=.70]{figures/iommu2}
\end{frame}

\begin{frame}{Memory access on modern systems}
\centering\includestandalone[scale=.70]{figures/iommu3}
\end{frame}

\begin{frame}{Memory access on modern systems}
\centering\includestandalone[scale=.70]{figures/iommu4}
\end{frame}

\begin{frame}{Memory access on modern systems}
\centering\includestandalone[scale=.70]{figures/iommu5}
\end{frame}

\begin{frame}{Memory access on modern systems}
\centering\includestandalone[scale=.70]{figures/iommu6}
\end{frame}

%\begin{frame}{Backup: Unprivileged user space drivers on Linux}
%\begin{itemize}
%\item[1.] Prepare the system as root
%\begin{itemize}
%\item[1.1.] Bind the device to the special \texttt{vfio} driver
%\item[1.2.] \texttt{chown} the special magic \texttt{vfio} device to your user
%\item[1.3.] Allow your user to lock some amount of memory via \texttt{ulimit}
%\end{itemize}
%\pause
%\item[2.] \texttt{mmap} the special magic \texttt{vfio} device
%\item[3.] Do some magic \texttt{ioctl} calls on the magic device
%\item[4.] Protected DMA memory can also be allocated via an \texttt{ioctl} call
%\item[5.] Use the device as usual, all accesses are now checked by the IOMMU
%\vspace{1em}
%\pause
%\item We have implemented this in our C driver, Rust is WIP
%\end{itemize}
%\end{frame}
%
%
%
%
%\begin{frame}{Why write a user space network driver?}
%\begin{itemize}
%\item Why not? It can be fun
%\item Maybe you need a quick \& dirty driver for a weird device?
%\item Maybe you need quick development cycles while playing around with a custom device
%\item Maybe you need some feature not supported by the original driver
%\end{itemize}
%\end{frame}
%
%
%\begin{frame}{Example: Hardware timestamping}
%\begin{itemize}
%\item Our latency measurement requires timestamps with nanosecond-level precision
%\item It also needs to handle millions of packets per second (we measured with $\approx$ 15\,Mpps)
%\item This usually requires special hardware (we've used NetFPGAs to do this in the past)
%\pause
%\vspace{1em}
%\item Some cheap off-the-shelf NICs can add a timestamp to all  incoming packets
%\item But none of the existing drivers support this feature :(
%\pause
%\item We just set some flags in the right registers and got precise timestamping for cheap
%\item We used a Xeon D embedded NIC capturing all packets via a fiber optic splitter before and after our device under test (precision $\approx$ \textpm 15\,ns)
%\end{itemize}
%\end{frame}

%\begin{frame}{(Maybe) network stack of the future}
%\end{frame}

%\begin{frame}{Conclusion: Check out our code}
%\centering \qrcode[height=3cm]{https://github.com/ixy-languages/ixy-languages}
%\begin{itemize}
%\item Meta-repository with links:\\\url{https://github.com/ixy-languages/ixy-languages}
%\item Drivers are simple: don't be afraid of them
%\item No kernel code needed :)
%\end{itemize}
%\centering \Huge Q \& A
%\end{frame}





\end{document}

